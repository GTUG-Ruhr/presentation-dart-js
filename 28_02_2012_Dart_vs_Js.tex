\documentclass[]{beamer}
\usetheme{Singapore}
\usepackage{listings}
\usepackage{color}
\definecolor{dkgreen}{rgb}{0,0.6,0}
\definecolor{gray}{rgb}{0.5,0.5,0.5}
\definecolor{mauve}{rgb}{0.58,0,0.82}
\lstset{ %
	language=vbscript,
	basicstyle=\scriptsize,
	numbers=left,
	numberstyle=\tiny\color{gray},
	stepnumber=2,
	numbersep=5pt,
	backgroundcolor=\color{white},      % choose the background color. You must add \usepackage{color}
	showspaces=false,               % show spaces adding particular underscores
	showstringspaces=false,         % underline spaces within strings
	showtabs=false,                 % show tabs within strings adding particular underscores
%	frame=single,                   % adds a frame around the code
	rulecolor=\color{black},        % if not set, the frame-color may be changed on line-breaks within not-black text (e.g. commens (green here))
	tabsize=2,                      % sets default tabsize to 2 spaces
	captionpos=b,                   % sets the caption-position to bottom
	breaklines=true,                % sets automatic line breaking
	breakatwhitespace=false,        % sets if automatic breaks should only happen at whitespace
	title=\lstname,                   % show the filename of files included with \lstinputlisting;
        	                          % also try caption instead of title
	keywordstyle=\color{blue},          % keyword style
	commentstyle=\color{dkgreen},       % comment style
	stringstyle=\color{mauve},         % string literal style
}
\lstset{ %
	language=html,
	basicstyle=\scriptsize,
	numbers=left,
	numberstyle=\tiny\color{gray},
	stepnumber=2,
	numbersep=5pt,
	backgroundcolor=\color{white},      % choose the background color. You must add \usepackage{color}
	showspaces=false,               % show spaces adding particular underscores
	showstringspaces=false,         % underline spaces within strings
	showtabs=false,                 % show tabs within strings adding particular underscores
%	frame=single,                   % adds a frame around the code
	rulecolor=\color{black},        % if not set, the frame-color may be changed on line-breaks within not-black text (e.g. commens (green here))
	tabsize=2,                      % sets default tabsize to 2 spaces
	captionpos=b,                   % sets the caption-position to bottom
	breaklines=true,                % sets automatic line breaking
	breakatwhitespace=false,        % sets if automatic breaks should only happen at whitespace
	title=\lstname,                   % show the filename of files included with \lstinputlisting;
        	                          % also try caption instead of title
	keywordstyle=\color{blue},          % keyword style
	commentstyle=\color{dkgreen},       % comment style
	stringstyle=\color{mauve},         % string literal style
}
\title{Dart Vs. Javascript}
\author{Matthias Kluth}
\date{\today}
\logo{\includegraphics[scale=0.01]{gtug-huge-flat}}
\begin{document}
\frame{\titlepage}
\frame{\frametitle{Table Of Contents}\tableofcontents}
\section{What is Javascript?}
\begin{frame}
\frametitle{What is Javascript?}
\begin{quote}
JavaScript is a prototype-based scripting language that is dynamic, weakly typed and has first-class functions. It is a multi-paradigm language, supporting object-oriented, imperative, and functional programming styles.
\end{quote}
- Wikipedia
\end{frame}
\section{What is Dart?}
\begin{frame}
\frametitle{What is Dart?}
\begin{quote}
Dart is intended to solve JavaScript's problems (which, according to a leaked memo, cannot be solved by evolving the language) while offering better performance, the ability "to be more easily tooled for large-scale projects" and better security features. Google will offer a cross compiler that compiles Dart to ECMAScript 3 on the fly, for compatibility with non-Dart browsers. There will also be a facility to convert typed Closure code to Dart. Google also plans to integrate a native VM into Chrome (there is a Chromium branch aiming to implement this) and encourage competitors to do the same with their browsers.
\end{quote}
- Wikipedia
\end{frame}
\section{Example - Hello, world JS}
\begin{frame}
\frametitle{Hello, world JS}
\lstinputlisting[language=html]{hello-world-js/index.html}
\end{frame}
\begin{frame}
\frametitle{Hello, world JS}
\lstinputlisting[language=vbscript]{hello-world-js/js/hello.js}
\end{frame}
\section{Example - Hello, world Dart}
\begin{frame}
\frametitle{Hello, world Dart}
\lstinputlisting[language=html]{hello-world-dart/index.html}
\end{frame}
\begin{frame}
\frametitle{Hello, world Dart}
\lstinputlisting[language=java]{hello-world-dart/dart/hello.dart}
\end{frame}
\section{Comparing}
\begin{frame}
\frametitle{Comparing HTML}
\begin{minipage}[t]{.45\textwidth}
\lstinputlisting[language=html]{hello-world-js/index.html}
\end{minipage}
\begin{minipage}[t]{.45\textwidth}
\lstinputlisting[language=html]{hello-world-dart/index.html}
\end{minipage}
\end{frame}
\begin{frame}
\frametitle{Comparing scripts}
\begin{minipage}[t]{.45\textwidth}
	\lstinputlisting[language=vbscript]{hello-world-js/js/hello.js}
\end{minipage}
\begin{minipage}[t]{.45\textwidth}
	\lstinputlisting[language=java]{hello-world-dart/dart/hello.dart}
\end{minipage}
\end{frame}
\section{Warning}
\begin{frame}
\frametitle{\color{red}Warning! Dartium only!}
At time of writing, pure Dart is only available in Dartium, a special Chromebuild. If you want to use it with other browsers, you have to compile it to Javascript. A compiler is included, but not yet optimized. Right now, a simple Hello, world example takes about 17000 lines when compiled to Javascript.
\end{frame}
\end{document}
